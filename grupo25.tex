\documentclass{article}

\usepackage{amsmath, amsthm, amssymb, amsfonts}
\usepackage{thmtools}
\usepackage{graphicx}
\usepackage{setspace}
\usepackage{geometry}
\usepackage{float}
\usepackage{hyperref}
\usepackage[utf8]{inputenc}
\usepackage[english]{babel}
\usepackage{framed}
\usepackage[dvipsnames]{xcolor}
\usepackage[most]{tcolorbox}
\usepackage{minted}

\usepackage{indentfirst}

\usepackage[export]{adjustbox} % Align images

\colorlet{LightGray}{White!90!Periwinkle}
\colorlet{LightOrange}{Orange!15}
\colorlet{LightGreen}{Green!15}

\newcommand{\HRule}[1]{\rule{\linewidth}{#1}}

\newtcbtheorem[auto counter,number within=section]{code}{Código}{
    colback=LightOrange!20,
    colframe=LightOrange,
    colbacktitle=LightOrange,
    fonttitle=\bfseries\color{black},
    boxed title style={size=small,colframe=LightOrange},
}{code}

\setstretch{1.2}
\geometry{
    textheight=22.5cm,
    textwidth=13.75cm,
    top=2.5cm,
    headheight=12pt,
    headsep=25pt,
    footskip=30pt
}

% ------------------------------------------------------------------------------

\begin{document}

% ------------------------------------------------------------------------------
% Cover Page and ToC
% ------------------------------------------------------------------------------
\begin{center}
    \begin{figure}
        \includegraphics[scale = 0.3, left]{img/IST_A.eps} % IST logo
        \end{figure}
    \LARGE{ \normalsize \textsc{} \\
    [2.0cm] 
    \LARGE{ \LARGE \textsc{Introdução à Arquitetura de Computadores}} \\
    [1cm]
    \LARGE{ \LARGE \textsc{LEIC IST-UL}} \\
    [1cm]
    \HRule{1.5pt} \\
    [0.4cm]
    \LARGE \textbf{\uppercase{Relatório - Projeto Beyond Mars}}
    \HRule{1.5pt}
    \\ [2.5cm]
    }
\end{center}

\begin{flushleft}
    \textbf{\LARGE Grupo 25:}
\end{flushleft}

\begin{center}
    \begin{minipage}{0.7\textwidth}
        \begin{flushleft}
            \large João Miguel Figueiredo Barros \\
            \large Guilherme Vaz Rocha \\
            \large Gabriel dos Reis Fonseca Castelo Ferreira 
        \end{flushleft}
    \end{minipage}%
    \begin{minipage}{0.3\textwidth}
        \begin{flushright}
            \large 106063\\
            \large 106171\\
            \large 107030
        \end{flushright}
    \end{minipage}
\end{center}

\begin{center}
    \vspace{4cm}
    \date \large \bf  2022/2023 -- 2º Semester, P4
\end{center}


\setcounter{page}{0}
\thispagestyle{empty}

\newpage

% ------------------------------------------------------------------------------
% Table of Contents
% ------------------------------------------------------------------------------


\tableofcontents
\newpage

% ------------------------------------------------------------------------------
% Content
% ------------------------------------------------------------------------------

\section{Detalhes}
Esta secção inclui detalhes sobre a abordagem tomada na implementação de algumas funcionalidades, cujo funcionamento não é óbvio e foi .

\subsection{Rotina valor\_teclado}
No processo de obtenção do valor da tecla premida a partir dos valores de linha e de coluna, é preciso converter primeiro estes valores de 1-2-4-8 para o seu bit ativo: 0-1-2-3.

\bigbreak

Olhemos então para a transformação que se efetua:

\begin{center}
    \begin{minipage}{1.5cm}
        \begin{center}
            X \\
            0001b 1 \\
            0010b 2 \\
            0100b 4 \\
            1000b 8
        \end{center}
    \end{minipage}
    \begin{minipage}{0.5cm}
        \begin{center}
            \bigbreak
            \textrightarrow \\
            \textrightarrow \\
            \textrightarrow \\
            \textrightarrow
        \end{center}
    \end{minipage}
    \begin{minipage}{1.5cm}
        \begin{center}
            Ret \\
            0000b 0 \\
            0001b 1 \\
            0010b 2 \\
            0011b 3
        \end{center}
    \end{minipage}
\end{center}

\bigbreak
Isto trata-se obviamente da operação \( Ret = log_{2} (X) \) (porque X é potência de 2) ou o número de \textit{trailing zeroes}, e é possível implementar essa operação para fazer a conversão, como indicado no guião do lab3, tendo um loop em que é incrementado um registo e feito SHR até sair um 1 pelo Carry. 

\bigbreak
Mas olhando mais perto, como são apenas estes valores, é possível fazer algo mais eficiente que não recorra ao uso de loops. Reparemos que de 1,2,4 para 0,1,2 apenas ocorre um SHR, mas um SHR de 8 é 4 e queremos 3. O que é possível fazer para apenas subtrair 1 quando acabamos com 4 é reparar que fazendo um SHR, 2 de 0,1,2,4 obtemos 0,0,0,1. Então o resultado do primeiro SHR seguido da subtração desse mesmo resultado após SHR, 2 é igual ao que a conversão quer devolver.

\bigbreak
O processo pode então ser descrito da seguinte forma:
\begin{center}
    \begin{minipage}{1.5cm}
        \begin{center}
            X \\
            Input
            0001b 1 \\
            0010b 2 \\
            0100b 4 \\
            1000b 8
        \end{center}
    \end{minipage}
    \begin{minipage}{0.5cm}
        \begin{center}
            \vspace{1.1cm}
            \textrightarrow \\
            \textrightarrow \\
            \textrightarrow \\
            \textrightarrow
        \end{center}
    \end{minipage}
    \begin{minipage}{1.5cm}
        \begin{center}
            Y \\
            SHR X, 1 \\
            0000b 0 \\
            0001b 1 \\
            0010b 2 \\
            0100b 4
        \end{center}
    \end{minipage}
    \begin{minipage}{0.5cm}
        \begin{center}
            \vspace{1.1cm}
            \textrightarrow \\
            \textrightarrow \\
            \textrightarrow \\
            \textrightarrow
        \end{center}
    \end{minipage}
    \begin{minipage}{1.5cm}
        \begin{center}
            Z \\
            SHR Y, 1 \\
            0000b 0 \\
            0000b 0 \\
            0000b 0 \\
            0001b 1
        \end{center}
    \end{minipage}
    \begin{minipage}{0.5cm}
        \begin{center}
            \vspace{1.1cm}
            \textrightarrow \\
            \textrightarrow \\
            \textrightarrow \\
            \textrightarrow
        \end{center}
    \end{minipage}
    \begin{minipage}{1.5cm}
        \begin{center}
            Ret \\
            Y-Z
            0000b 0 \\
            0001b 1 \\
            0010b 2 \\
            0011b 3
        \end{center}
    \end{minipage}
\end{center}

\bigbreak

Na linguagem de Assembly P4, sendo RX um registo onde least significant nibble é X e o resto 0, ficam apenas 4 instruções:

\bigbreak

\begin{code}{Conversão 1,2,4,8 para 0,1,2,3}{}
    \begin{minted}{gas}
        SHR RX, 1   ; RX = Y
        MOV R2, RX  ; R2 = Y
        SHR R2, 2   ; R2 = Z
        SUB RX, R2  ; RX = Y-Z
    \end{minted}
\end{code}

\newpage

% ----------------------------------------------------------------------
% Cover
% ----------------------------------------------------------------------

\end{document}
